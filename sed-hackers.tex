% Author: Victor Terron (c) 2013
% Email: `echo vt2rron1iaa32s | tr 132 @.e`
% License: CC BY-SA 4.0

\documentclass[14pt]{beamer}

\usepackage[utf8]{inputenc}
\usepackage{amsmath}
\usepackage{amsfonts}
\usepackage{amssymb}
\usepackage[spanish]{babel}
\usepackage{listings}
\usepackage{eurosans}
\usepackage{bold-extra}
\usepackage{caption}
\usepackage{marvosym}

\usetheme{Copenhagen}
\useoutertheme{infolines}
\setbeamercovered{dynamic}
\setbeamertemplate{navigation symbols}{} % remove navigation symbols

% A Nice Title Page for Beamer Presentations
% http://github.com/dfalster/BeamerTitleSlide
\usepackage{tikz}
\usepackage[framemethod=tikz]{mdframed}

% Define a new mdframed environment
\newmdenv[tikzsetting={draw=black,fill=white,fill opacity=0.8},
  backgroundcolor=none,leftmargin=0,rightmargin=0, innertopmargin=4pt,
  skipbelow=\baselineskip, skipabove=\baselineskip]{TitleBox}

% http://www.colourlovers.com/palette/580974/Adrift_in_Dreams
\definecolor{sail-green}{HTML}{3B8686}

\lstset{basicstyle=\ttfamily,language=python}

\newcommand{\ShowCurrentSection}[0]{
  \AtBeginSection[]{
  \begin{frame}
      \frametitle{Índice}
      \begin{columns}
        \column{.4\textwidth}
          \begin{flushright}
            \includegraphics[width=3cm]{pics/torvalds-to-nvidia.jpg}
          \end{flushright}
        \column{.5\textwidth}
          \tableofcontents[currentsection]
      \end{columns}
  \end{frame}}
}

\newcommand{\WebLink}[2]{
  \href{#1}{\structure{\PointingHand~\color{sail-green}{#2}}}
}

\title{¡Sed Hackers!}
\author{Víctor Terrón}
\date{19 de febrero de 2016}
\institute{@pyctor}

\begin{document}

{
\usebackgroundtemplate{\includegraphics[height=1.0\paperheight]{pics/ninjas.jpg}}

\begin{frame}[plain]
  \vspace{6.25cm}
  \begin{TitleBox}
    {\LARGE \inserttitle} \\
    {\small \insertauthor \enspace --- \thinspace \url{https://twitter.com/pyctor}}
  \end{TitleBox}
\end{frame}
}

\section{Introducción}
\begin{frame}{}
\begin{block}{}
    \centering \Large Las tres salidas después de la \structure{\tt ETSIIT}
\end{block}
\end{frame}

\begin{frame}{Tierra}
  \begin{center}
    \includegraphics[width=0.6\textwidth]{pics/thelma-louise-cliff.jpg}
  \end{center}
\end{frame}

\begin{frame}{Mar}
  \begin{center}
    \includegraphics[width=0.6\textwidth]{pics/ship-sinking.jpg}
  \end{center}
\end{frame}

\begin{frame}{Aire}
  \begin{center}
    \includegraphics[width=0.6\textwidth]{pics/plane-crashing.jpg}
  \end{center}
\end{frame}

\begin{frame}{}
\begin{block}{}
    \centering \Large El destino de muchos: \structure{\tt consultoría}
\end{block}
\begin{figure}
  \centering
  \includegraphics[width=0.4\textwidth]{pics/ben-hur-rowing.jpg}
  \caption*{Accenture, Northgate Arinso y demás cárnicas}
\end{figure}
\end{frame}

\begin{frame}{ETSIIT: A New Hope}
  \begin{itemize}
    \item Hay vida más allá de \structure{Arinso}
    \item Hay países donde la gente joven encuentra trabajo
    \item El ejemplo más evidente es \structure{Silicon Valley} \\
      {\small Empresas líder en el sector, proyectos muy interesantes
        y unas condiciones laborales impensables aquí en España}
    \item Todos querríamos \structure{trabajar} en una empresa así
    \item ¿Cuál es el \structure{perfil} de aquellos que lo consiguen?
  \end{itemize}
  \begin{center} \Large \bf ¡Todos son hackers!\end{center}
\end{frame}


\begin{frame}{Hackers}
    \begin{center}
      Hackers, por supuesto, en su \structure{verdadero significado}
    \end{center}

    \begin{block}{\footnotesize Definición según Wikipedia}
      \centering \footnotesize A person who enjoys \structure{exploring the
        limits of what is possible}, in a spirit of playful
      cleverness. They may also heavily modify software or hardware of
      their own computer system. It includes building, rebuilding,
      modifying, and creating software, or anything else, either to
      \structure{make it better} or faster or to give it added
      features or to \structure{make it do something it was never
        intended to do}.
    \end{block}

  \begin{itemize}
    \item \WebLink{http://www.catb.org/esr/faqs/hacker-howto.html}
                  {How To Become A Hacker}, de Eric Raymond
    \item A todos los hackers les \structure{apasiona} la programación
  \end{itemize}
\end{frame}


\begin{frame}{}
  \vspace{2cm}
  \begin{alertblock}{}
    \centering Porque a todos nos apasiona programar... \textbf{¿verdad?}
  \end{alertblock}
  \begin{center}
    \includegraphics[width=5cm]{pics/you_dont_say.png}
  \end{center}
\end{frame}

\begin{frame}{}
  \begin{center}
    \includegraphics[width=0.8\textwidth]{pics/nothing-to-do-here.jpg}
  \end{center}
\end{frame}

\begin{frame}{Hacker = programador}
\begin{itemize}
\item La programación es \structure{fundamental} en este gremio
\item Idea \structure{absurda}: \emph{``Los ingenieros no programan''}
\item No sólo lo hacen, sino que son \structure{mejores} que nadie
\item La alternativa es ser... un \structure{powerpoinista}.
\item \WebLink{http://www.alfredodehoces.com/fuckowski-on-line}
              {Fuckowski, memorias de un ingeniero}
\end{itemize}
\end{frame}

\begin{frame}{}
  \begin{center}
    \includegraphics[width=0.6\textwidth]{pics/the-oatmeal-real-geeks.jpg}
  \end{center}
\end{frame}

\begin{frame}{Startups}
  \begin{block}{} \centering
    Es posible que ya hayan empezado a bombardearos con la idea, y en los
    años próximos irá a más, de que hay que \structure{\emph{emprender}}.
    La fórmula tiene algunas variaciones, pero siempre gira en torno a
    la idea de que unos cuantos recién graduados podríais montar una
    empresa de base tecnológica para abriros un hueco en el mundo.
  \end{block}

  \begin{itemize} \small
    \item Películas como \structure{The Social Network} (2010)
    \item Ayudas a jóvenes emprendedores
    \item \structure{The Lean Startup}, de Eric Ries
  \end{itemize}
\end{frame}

\begin{frame}{Startups}
    \begin{block}{}
    \centering \Large Expectativas
  \end{block}

  \begin{figure}
    \centering
    \includegraphics[width=0.75\paperwidth]{pics/scrooge-mcduck.jpg}
  \end{figure}
\end{frame}

\begin{frame}{Startups}
    \begin{alertblock}{}
    \centering \Large Realidad
  \end{alertblock}

  \begin{columns}
    \column{.5\textwidth}
      \includegraphics[width=0.4\paperwidth]{pics/code-for-food-one.jpg}
    \column{.5\textwidth}
      \includegraphics[width=0.4\paperwidth]{pics/code-for-food-two.jpg}
  \end{columns}
\end{frame}

\begin{frame}{Startups}
  \begin{alertblock}{} \centering

    Lo más probable es que acabéis igual de pobres que antes de
    empezar, pero \structure{arruinados} y quizás habiendo arrastrado
    a familiares y amigos.
  \end{alertblock}

  \begin{itemize} \small
    \item No seáis \structure{autónomos kamikaze}.
    \item Una empresa se monta sólo cuando se sabe \structure{muy
      bien} lo que se está haciendo y se tiene \structure{mucha
      experiencia}, e incluso así lo más habitual es que salga mal.
    \item El \structure{~80\%} de las empresas \structure{quiebran} en
      los primeros cinco años.
    \item Y ahora la tendencia de consumo y financiación es
      \structure{negativa}.
    \item No ha de hacerse jamás como alternativa al \structure{desempleo}.
    \item El mensaje se difunde para sacaros vuestros ahorros.
  \end{itemize}
\end{frame}

\begin{frame}{}
\begin{itemize}
  \item Existen empresas como Valve, Twitter o GitHub
  \item Sin \structure{horarios} ni código de \structure{vestimenta}
  \item El \WebLink{http://www.glassdoor.com/Salary/Twitter-Salaries-E100569.htm}
                   {salario medio}
        en Twitter es de \structure{\$127,486} (¡!)
  \item Valve es un paradigma de \structure{organización horizontal}
  \item \WebLink{http://www.valvesoftware.com/company/Valve_Handbook_LowRes.pdf}
                {Manual del nuevo empleado de Valve}
  \item Aspirad a trabajar en un sitio así
  \item Aspirad a ser \structure{hackers}
\end{itemize}
\end{frame}

\ShowCurrentSection

\setbeamerfont{itemize/enumerate body}{size=\footnotesize}

\section{Y éste quién es}
\begin{frame}{Quién soy}
\begin{itemize}
  \item Víctor Terrón
  \item \url{http://www.github.com/vterron}
  \item Ingeniería Informática (2003-2009), Universidad de Granada
  \item Intercambio en Irlanda (\structure{Erasmus}) y EE.UU. (\structure{programa propio})
  \item Hasta 2015 trabajé en el Instituto de Astrofísica de Andalucía (CSIC)
  \item Desarrollaba software para \structure{instrumentos astronómicos}
  \item En ocasiones era \structure{operador de telescopio} en Calar Alto
  \item Desde 2009, \structure{semanas después} de terminar la carrera
  \item Ahora trabajo en Google UK como \structure{Site Reliability Engineer}
\end{itemize}
  \begin{alertblock}{\centering \scriptsize Aclaración obligada}
    \centering \scriptsize Las opiniones aquí vertidas son
    exclusivamente mías, y no representan la opinión de ningún
    empleador pasado, presente o futuro. Y son sólo eso — ¡opiniones!
  \end{alertblock}
\end{frame}

\begin{frame}{1.23m CAHA}
  \begin{center}
    \includegraphics[height=0.8\textheight]{pics/123m_CAHA_1.jpg}
  \end{center}
\end{frame}

\begin{frame}{1.23m CAHA}
  \begin{center}
    \includegraphics[height=0.8\textheight]{pics/123m_CAHA_2.jpg}
  \end{center}
\end{frame}

\setbeamerfont{itemize/enumerate body}{size=\small}

\begin{frame}{¿Y eso?}
\begin{itemize}
  \item Mi paso por el IAA fue \structure{el primer y más importante
    paso en mi carrera profesional}, y lo que hizo posible que seis
    años después me fuera a Londres.
  \item La pregunta es \structure{por qué me cogieron a mí}, y no a otro
  \item En última instancia, buscaban dos cosas en un candidato:
     \begin{itemize}
       \item Que supiera de \structure{GNU/Linux}
       \item Y hablara \structure{inglés}
     \end{itemize}
  \item Y ésas eran básicamente las únicas dos cosas que yo sabía
\end{itemize}
  \begin{alertblock}{\centering Primer Axioma}
    \centering El expediente sólo sirve para que te den becas
  \end{alertblock}
\end{frame}

\begin{frame}{Joel Spolsky}

  \begin{figure}
    \centering
    \includegraphics[height=0.7\textheight]{pics/smart-and-gets-things-done.jpg}
    \caption*{
      \small
      \WebLink{http://www.joelonsoftware.com/items/2007/06/05.html}
              {Smart and Gets Things Done}}
  \end{figure}

\end{frame}

\setbeamerfont{itemize/enumerate body}{size=\footnotesize}

\begin{frame}{}
  \begin{block}{}\centering
    \Large \structure{GNU/Linux}
  \end{block}
  \begin{itemize}
    \item El manejo de la línea de comandos es \structure{esencial}
    \item La curva de aprendizaje es suave (es decir, muy difícil)
    \item No será \structure{cómodo} al principio, ni agradable
    \item \emph{¿No acabaría antes....?} Respuesta: \structure{sí}
    \item Pero aprenderéis muchísimo — incontables conceptos
    \item La abstracción de las GUI nos \structure{limitan intelectualmente}
    \item No seáis otra \structure{Generación XP}
    \item \structure{In the Beginning... Was the Command Line}, de  Neal Stephenson
  \end{itemize}
\end{frame}

\begin{frame}{}
  \begin{block}{}\centering
    \Large \structure{Inglés}
  \end{block}
  \begin{itemize}
    \item El 99\% del material existente \structure{está en la lingua franca}
    \item El 1\% restante son traducciones del inglés (por ejemplo, Wikipedia)
    \item Las de arriba son cifras inventadas, pero captáis el mensaje
    \item Estudiad como sea al menos un año en \structure{un país de habla inglesa}
    \item A ser posible \structure{el último curso} (para no volver después)
    \item Empezad a ahorrar ya si hace falta, aunque \structure{tampoco necesitáis tanto}
    \item Yo gasté \EUR{8,500} en \structure{un curso entero} en California
    \item Necesitáis un título: \structure{Certificate of Advanced English}
    \item El \structure{First} está bien cuando tienes quince años
  \end{itemize}
\end{frame}

\section{¿Y vosotros?}
\begin{frame}{Los años que os quedan}
  \begin{itemize}
    \item Tenéis por delante unos años \structure{bastante feos} en la
      Universidad.
    \item Los profesores buenos con \structure{escasísimos}, y muy valiosos.
    \item Los mediocres o directamente \structure{inútiles abundan}, y
      se reproducen a una velocidad asombrosa. Parecen destinados a
      dominar el mundo.
    \item Consejo: centrad vuestros esfuerzos en los pocos docentes e
      investigadores que \structure{merecen la pena}.
    \item El mundo ya está lleno de gente que \structure{se limitó a
      aprobar asignaturas}, incluso con buena nota.
  \end{itemize}

  \begin{alertblock}{\centering Segundo Axioma}
    \centering No hay asignaturas difíciles, sólo malos profesores
  \end{alertblock}
\end{frame}

\begin{frame}{Carpe Diem}
  \begin{itemize}
    \item No quiero sonar como un viejo, ¡pero \structure{aprovechad
      el tiempo}!
    \item WoW, LoL, Facebook, Tuenti, Cuánto Cabrón o Series Yonkis
    \item Los que dediquéis todo ese tiempo a esfuerzos creativos
      seréis \structure{expertos con varios años de experiencia} para
      cuando obtengáis el título.
    \item El resto empezaréis a aprender en serio \structure{sólo
      entonces}, y estaréis como mínimo varios años por detrás de los
      que hiceron algo más que ir a clase, prácticas y exámenes.
    \item Todos los hackers se caracterizan por \structure{aprovechar
      muy bien el tiempo}. Hay tiempo para todos los proyectos que os
      propongáis.
    \item No gastéis esfuerzos en \structure{conocimientos inútiles}
      como saberos al dedillo cuáles son los últimos modelos en
      tarjetas gráficas. Dentro de 50 años se \structure{seguirá
        programando en Fortran y C}, pero no habrá APIs para Facebook.
    \item ¡Vamos a hacer cosas!
  \end{itemize}
\end{frame}

\begin{frame}{Doctorado}
  \begin{itemize}
    \item Un \structure{máster} una vez acabéis \emph{puede} ser una
      buena idea.
    \item El \structure{doctorado} —el 99\% de las ocasiones—
      \structure{no}.
    \item No hagáis jamás un doctorado \structure{por inercia}, o
      porque toca.
    \item Tenéis que estar muy, muy, muy seguros antes de empezar uno.
    \item Tened un cuenta el \structure{coste de oportunidad} de todos
      esos años.
    \item El doctorado \structure{combina lo peor de los dos mundos}:
      la servidumbre académica con unas condiciones laborales propias
      de campesino del siglo XIII.
    \item Idea: probad \structure{unos años en la industria} y volved
      después al mundo académico con las ideas claras y el Kung Fu
      afilado.
  \end{itemize}

  \begin{block}{\centering \small Lectura obligada}
    \centering \small
      \WebLink{http://matt.might.net/articles/phd-school-in-pictures/}
              {The illustrated guide to a Ph.D.}
  \end{block}

\end{frame}

\begin{frame}{Randy Pausch (1960---2008)}

  \begin{center}
    \WebLink{https://www.youtube.com/watch?v=ji5_MqicxSo}
            {The Last Lecture: Achieving Your Childhood Dreams}

            \CrossedBox \\
    \WebLink{https://www.youtube.com/watch?v=oTugjssqOT0}
            {Time Management}
  \end{center}

  \begin{figure}
    \centering
    \includegraphics[width=0.6\textwidth]{pics/randy-pausch.jpg}
    \caption*{\small Si sólo vais a hacer click en dos enlaces, \structure{que sean estos}}
  \end{figure}

\end{frame}

\setbeamerfont{itemize/enumerate body}{size=\normalsize}

\section{How to Become a Ninja}

\ShowCurrentSection

\begin{frame}{Vale... ¿pero cómo?}
  \begin{itemize}
    \item Empezad usando GNU/Linux como \structure{único sistema
      operativo}, para todo
    \item Aprended sólidamente los fundamentos de la programación, y
      de aquí a cinco años proponéos dominar {\bf al menos} tres
      lenguajes:
         \begin{itemize}
           \item Uno de \structure{bajo} nivel (C o C++)
           \item Uno de \structure{alto} (Python o Perl)
           \item Y uno \structure{\emph{raro}} (Lisp o Erlang)
         \end{itemize}
    \item No es nunca la sintaxis, sino los \structure{paradigmas}.
    \item Expresarse \structure{con fluidez} en inglés es esencial.
    \item No olvidéis el \structure{perfil blando}: música, artes
      marciales.
    \item Nunca preguntéis \emph{``¿Y esto para qué sirve?''}
   \end{itemize}
\end{frame}

\setbeamerfont{itemize/enumerate body}{size=\small}

\begin{frame}{Software Libre}

   \begin{block}{}\centering
    \normalsize Involucraos en un proyecto de software libre.
   \end{block}

   \begin{itemize}
     \item Por más que algunos profesores que tendréis discrepen
     \item No hay nada que \structure{impresione} más en un currículum
     \item Encontrad \structure{un proyecto que os guste}, y empezad poco a poco
     \item Parches muy pequeños al principio
     \item Podéis empezar con \structure{traducciones}, si lo preferís
     \item Launchpad (Ubuntu) o GitHub
   \end{itemize}
\end{frame}

\begin{frame}{Titulitis}
  \begin{itemize}
    \item Nunca nadie fue \emph{``Ingeniero superior''}
    \item Incluso \emph{``Ingeniería''} a secas son palabras mayores
    \item No lo planteéis jamás como un \structure{Graduados} vs
      \structure{FPs}
    \item Telecomunicaciones mola porque aprenden más \structure{Física}
    \item El título es sólo \structure{un trozo de papel}
    \item Tenéis la obligación moral de ser \structure{humildes}
  \end{itemize}

  \begin{alertblock}{\centering Tercer Axioma}
    \centering Los de FP probablemente os dan mil vueltas
  \end{alertblock}
\end{frame}



\begin{frame}{git \textgreater svn}

  \begin{alertblock}{\centering Primer Mandamiento}
    \centering ¡No uséis Subversion!
  \end{alertblock}

  \begin{itemize} \itemsep0em
    \item Usad sistemas de control de versiones \structure{distribuidos}
    \item Mercurial o Git, ya es una cuestión de gustos
  \end{itemize}

  \begin{block}{}\centering
    \scriptsize For the first 10 years of kernel maintenance, we
    literally used tarballs and patches, \structure{which is a much
      superior source control management system than CVS is}, but I
    did end up using CVS for 7 years at a commercial company and I
    hate it with a passion. When I say I hate CVS with a passion, I
    have to also say that if there are any SVN (Subversion) users in
    the audience, you might want to leave. Because my hatred of CVS
    has meant that \structure{I see Subversion as being the most
      pointless project ever started}. The slogan of Subversion for a
    while was \emph{CVS done right}, or something like that, and if
    you start with that kind of slogan, there's nowhere you can
    go. There is no way to do CVS right.
  \end{block}

  \vspace{-0.5cm}

  \begin{center}
    \WebLink{https://www.youtube.com/watch?v=4XpnKHJAok8}
            {\small Linus Torvalds on Git (2007)}
  \end{center}
\end{frame}

\begin{frame}{Dijkstra}
   \begin{block}{}\centering
   \normalsize The teaching of BASIC should be rated as
   a criminal offence: it mutilates the mind beyond recovery.
   \end{block}
  \vspace{-0.5cm}
  \begin{center} \scriptsize Edsger W. Dijkstra (1984) \end{center}

  \begin{figure}
    \centering
    \includegraphics[width=0.45\textwidth]{pics/svn-reeducation.jpg}
    \caption*{\url{http://hginit.com/}}
  \end{figure}
\end{frame}

\begin{frame}{Emacs}

  \begin{block}{\scriptsize Neal Stephenson} \centering
    \scriptsize I use Emacs, which might be thought of as a thermonuclear word processor.
  \end{block}

  \begin{block}{\scriptsize Eric S. Raymond} \centering
    \scriptsize Emacs is undoubtedly the most powerful programmer's editor in
    existence. It's a big, feature-laden program with a great deal of
    flexibility and customizability. [...] Emacs has an entire
    programming language inside it that can be used to write
    arbitrarily powerful editor functions.
  \end{block}

  \begin{itemize} \itemsep0em
    \item IDEs como \structure{Eclipse} son cómodas pero simplifican demasiado
    \item Aprended a operar a mano \structure{antes de usar una calculadora}
    \item \structure{Real Programmers} use Emacs! — \url{https://xkcd.com/378/}
  \end{itemize}

\end{frame}

\begin{frame}{rsync}

  \begin{itemize} \itemsep0em
    \item Herramienta \structure{fundamental} para la sincronización
      de directorios
    \item Hace copias en local o remotas (vía \structure{SSH})
    \item Transfiere sólo los archivos que se han \structure{modificado}
    \item Y de éstos sólo las partes diferentes
      (\structure{compresión delta})
    \item Usa \structure{checksum} para verificar que las copias son
      \structure{idénticas}
  \end{itemize}

  \begin{alertblock}{\centering \footnotesize Escenarios habituales}
    \begin{itemize} \itemsep0em
      \footnotesize
      \item Uso básico: para copias de seguridad
      \item Uso avanzado: periódicas, usando \structure{cron}
      \item Uso \structure{hacker}: periódicas e incrementales
    \end{itemize}
  \end{alertblock}

  \begin{block}{}
    \centering \small
    El \structure{Time Machine} de Apple es rsync con interfaz gráfica
  \end{block}

\end{frame}

\begin{frame}{Dos casos}

  \begin{exampleblock}{Escenario A}\centering
   \small Abrirle a tu hámster la puerta de su jaula a mano
   \end{exampleblock}

  \begin{block}{Escenario B} \centering

    \small Automatizar la apertura de la jaula con
    \structure{Arduino}, utilizando un pequeño servomotor que se
    activa a una hora determinada. Programar en \structure{Python} un
    sistema de reconocimiento de imágenes, ejecutándose en una
    \structure{Raspberry Pi}, que detecte cuándo ha vuelto dentro y
    cierre la puerta. Monitorizar la actividad del hámster y, en caso
    de detectar que hoy no ha salido de la jaula, usar la
    \structure{API de Twilio} para enviarnos aviso.
   \end{block}

  \begin{alertblock}{\centering Cuarto Axioma}
    \centering Difícil es más divertido
  \end{alertblock}

\end{frame}

\begin{frame}{GitHub}
  \begin{itemize}
    \item GitHub (o equivalente) es tu nuevo currículum
    \item Muestra de forma transparente \structure{qué has hecho, cómo y cuándo}
    \item Permite evaluar la \structure{calidad de tu código y contribuciones}
    \item Para las empresas buenas, esto es lo \structure{único que importa}
  \end{itemize}

  \begin{center}
    \includegraphics[width=6cm]{pics/github-logo.png}
  \end{center}
\end{frame}

\begin{frame}{GitHub}

  \begin{block}{\scriptsize} \centering
    \Large Colgad en GitHub {\bf todo} lo que hagáis
  \end{block}

  \begin{itemize}
    \item Desde \structure{prácticas} a ficheros de configuración rc
    \item Siempre hay alguien a quien le serán útiles
    \item Devolved a la comunidad parte del esfuerzo
    \item Sed \structure{creadores} de contenidos, no sólo consumidores
  \end{itemize}

\end{frame}

\begin{frame}{Videojuegos}
  \centering \small
  Si os interesa el mundo de los videojuegos, tenéis que aprender

  \begin{alertblock}{}
    \centering \LARGE C++ a muerte
  \end{alertblock}

  \begin{itemize} \small
    \itemsep0em
    \item Es el lenguaje de verdad en este gremio.
    \item Programación a \structure{bajo nivel} y \structure{muy
      optimizada}.
    \item Esto tiene la ventaja de que con ese perfil podréis saltar
      a \structure{cualquier otro sector}, ya que seréis grandes
      programadores, muy \structure{todoterreno}.
    \item Necesitaréis también \structure{matemáticas}. Muchas matemáticas.
  \end{itemize}

  \begin{block}{\centering \small Lectura obligada}
    \centering \small Why your games are made by childless, 31 year
    old white men, and how one studio is fighting back
    \WebLink{https://web.archive.org/web/20131211193933/http://penny-arcade.com/report/article/why-your-games-are-made-by-childless-31-year-old-white-men-and-how-one-stud}
            {Internet Archive}
  \end{block}
\end{frame}

\begin{frame}{Certificaciones}

  \begin{block}{}
    \centering
    Por norma general, las certificaciones relacionadas con la
    programación o los sistemas Unix son prácticamente
    \structure{inútiles}, y un desperdicio de \structure{tiempo} y
    \structure{dinero}.
  \end{block}

  \begin{itemize}
    \item ¡Más titulitis!
    \item Por ejemplo, las del
      \WebLink{https://www.lpi.org/linux-certifications}
              {Linux Professional Institute}
    \item Aún más divertido:
      \WebLink{http://www.oreillyschool.com/certificate-programs/python-programming/}
              {Python Programming Certificate}
    \item Hay excepciones, como (quizás) las de
      \WebLink{http://cisco.com/web/learning/certifications/entry/ccent/index.html}
              {Cisco}
    \item Nada que objetar si lo veis como una forma más de aprender.
    \item No olvidéis plataformas como Coursera o Udacity.
  \end{itemize}

\end{frame}

\begin{frame}{}
  \begin{columns}
    \column{.6\textwidth}
    \begin{block}{}\centering\Large\bf
      ¿La mejor forma de \structure{aprender}?
    \end{block}
    \begin{center}
      \structure{\huge Hacer cosas \emph{guays} \textbf{sin pensar}}
    \end{center}
    \column{.4\textwidth}
    \includegraphics[width=.9\textwidth]{pics/be-cool.jpg}
  \end{columns}
\end{frame}

\begin{frame}{Arduino}
  \begin{center}
  \includegraphics[width=.45\textwidth]{pics/arduino.jpg}
  \end{center}
  \vspace{-0.5cm}
  \begin{block}{}\centering
    Plataforma de hardware libre, basada en una placa con un
    microcontrolador y un entorno de desarrollo
  \end{block}
\end{frame}

\begin{frame}{Arduino: Tanque}
  \begin{figure}
    \centering
    \includegraphics[width=0.8\textwidth]{pics/arduino-tank.jpg}
    \caption*{\small \url{http://beatty-robotics.com/mechatronic-tank/}}
  \end{figure}
\end{frame}

\begin{frame}{Arduino: Araña}
  \begin{figure}
    \centering
    \includegraphics[width=0.7\textwidth]{pics/arduino-spider.jpg}
    \caption*{\small \url{http://www.flickr.com/photos/wizard23/3911240094/}}
  \end{figure}
\end{frame}

\begin{frame}{Arduino: Cuadricóptero}
  \begin{figure}
    \centering
    \includegraphics[width=0.7\textwidth]{pics/arduino-quadcopter.jpg}
    \caption*{\small \url{http://aeroquad.com/}}
  \end{figure}
\end{frame}

\begin{frame}{Raspberry Pi}
  \begin{center}
  \includegraphics[width=.45\textwidth]{pics/raspberry-pi.jpg}
  \end{center}
  \vspace{-0.5cm}
  \begin{block}{}\centering
    Una placa de ordenador de bajo coste del tamaño de una tarjeta de
    crédito. Se puede instalar Debian (Raspbian) y tiene salida 1080p
    HDTV por HDMI.
  \end{block}
\end{frame}

\begin{frame}{Raspberry Pi: Servidor Torrent}
  \begin{figure}
    \centering
    \includegraphics[width=0.5\textwidth]{pics/raspberry-pi-torrent-server.jpg}
    \caption*{\centering \small
      \url{http://eiosifidis.blogspot.com.es/2013/03/use-raspberry-pi-as-torrent-download.html}}
  \end{figure}
\end{frame}

\begin{frame}{Raspberry Pi: Luces de escritorio}
  \begin{figure}
    \centering
    \includegraphics[width=0.6\textwidth]{pics/raspberry-pi-color-my-desk.jpg}
    \caption*{\centering \small
      \url{http://makezine.com/raspberry-pi-design-contest/rpidcg_005_color-my-desk/}}
  \end{figure}
\end{frame}

\begin{frame}{Raspberry Pi: Clúster de 64 nodos}
  \begin{figure}
    \centering
    \includegraphics[width=0.7\textwidth]{pics/raspberry-pi-supercomputer.jpg}
    \caption*{\small \url{http://www.southampton.ac.uk/~sjc/raspberrypi/}}
  \end{figure}
\end{frame}

\section{Conclusión}

\ShowCurrentSection

\begin{frame}{Conclusión}
    \begin{block}{} \centering
      \normalsize El mundo es un lugar \structure{fantástico}, lleno
      de gente \structure{increíble} que trabaja en proyectos
      \structure{interesantes}. No seáis una gota más en un océano de
      mediocridad. Entregaos en cuerpo y alma a aquello que os
      apasione.
    \end{block}

    \begin{itemize} \itemsep0em
      \item Sólo si os gusta algo podréis llegar a ser
        \structure{realmente buenos}
      \item El futuro pertenece a los {\bf \structure{frikis}} (los de verdad)
      \item \WebLink{https://vimeo.com/65666763}
                    {What if Money Did not Matter?}, de Alan Watts
      \item \WebLink{https://www.youtube.com/watch?v=rAn4gZCd4HY}
                    {Everybody's Free To Wear Sunscreen}, de Baz Luhrmann
    \end{itemize}
\end{frame}


\end{document}

